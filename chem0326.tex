\documentclass[border=3pt,varwidth=110mm]{standalone}
%standard width: 70 mm
\input{preamble}
\excludecomment{answer}
%\includecomment{answer}

\begin{document}
Consider the following conversion:
\begin{center}
\schemestart
\chemfig{*5(--(-[:-30]OH)(-[:30])---)}
\arrow{->[Step (I)]}
\parbox{2cm}{\centering intermediate\\compound}
\arrow{->[Step (II)]}
\chemfig{*5(--(-)---)}
\schemestop
\end{center}
Which of the following combinations can achieve the above conversion?

\begin{center}
\begin{tabular}{ccc}
   & \underline{Reagent used in Step (I)} & \underline{Reagent used in Step (II)} \\[1em]
A. & concentrated \ch{H2SO4}  & \ch{LiAlH4} \\[1em]
B. & acidified \ch{KMnO4}  & \ch{LiAlH4} \\[1em]
C. & concentrated \ch{H2SO4}  & \ch{H2/Ni} \\[1em]
D. & acidified \ch{KMnO4}  & \ch{H2/Ni} \\[1em]
\end{tabular}
\end{center}

\begin{answer}
\hrulefill\par
\textbf{Ans: C}

\end{answer}
\end{document}
