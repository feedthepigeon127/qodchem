\documentclass[border=3pt,varwidth=110mm]{standalone}
%standard width: 70 mm
\usepackage[UKenglish]{babel}

\usepackage[utf8]{inputenc}
\usepackage[T1]{fontenc}
\usepackage{libertine}
\usepackage[libertine]{newtxmath}

\usepackage[autostyle]{csquotes}
\usepackage[free-standing-units, binary-units]{siunitx}
\sisetup{%
locale=UK,
range-phrase=--,
table-figures-decimal=2,
table-auto-round=true,
}
\DeclareSIUnit\Molar{\textsc{m}}

\usepackage{chemmacros, chemfig}
\setchemfig{atom sep=2em}

\usepackage{amsmath, amsfonts, amssymb}
\usepackage{textcomp}
\usepackage{bm}
\usepackage{xfrac}

\usepackage{varwidth}
\usepackage{comment}
\makeatletter
\g@addto@macro{\@parboxrestore}{\parskip=\baselineskip}
\makeatother

\usepackage[inline]{asymptote}

\newcounter{choice}
\renewcommand\thechoice{\Alph{choice}}
\newcommand\choicelabel{\thechoice.}

\newenvironment{choices}%
  {\list{\choicelabel}%
     {\usecounter{choice}\def\makelabel##1{\hss\llap{##1}}%
       \settowidth{\leftmargin}{W.\hskip\labelsep\hskip 2.5em}%
       \def\choice{%
         \item
       } % choice
       \labelwidth\leftmargin\advance\labelwidth-\labelsep
       \topsep=0pt
       \partopsep=0pt
     }%
  }%
  {\endlist}

\newenvironment{oneparchoices}%
  {%
    \setcounter{choice}{0}%
    \def\choice{%
      \refstepcounter{choice}%
      \ifnum\value{choice}>1\relax
        \penalty -50\hskip 1em plus 1em\relax
      \fi
      \choicelabel
      \nobreak\enskip
    }% choice
    % If we're continuing the paragraph containing the question,
    % then leave a bit of space before the first choice:
    \ifvmode\else\enskip\fi
    \ignorespaces
  }%
  {}
\excludecomment{answer}
%\includecomment{answer}

\begin{document}
Consider the following conversion:
\begin{center}
\schemestart
\chemfig{*5(--(-[:-30]OH)(-[:30])---)}
\arrow{->[Step (I)]}
\parbox{2cm}{\centering intermediate\\compound}
\arrow{->[Step (II)]}
\chemfig{*5(--(-)---)}
\schemestop
\end{center}
Which of the following combinations can achieve the above conversion?

\begin{center}
\begin{tabular}{ccc}
   & \underline{Reagent used in Step (I)} & \underline{Reagent used in Step (II)} \\[1em]
A. & concentrated \ch{H2SO4}  & \ch{LiAlH4} \\[1em]
B. & acidified \ch{KMnO4}  & \ch{LiAlH4} \\[1em]
C. & concentrated \ch{H2SO4}  & \ch{H2/Ni} \\[1em]
D. & acidified \ch{KMnO4}  & \ch{H2/Ni} \\[1em]
\end{tabular}
\end{center}

\begin{answer}
\hrulefill\par
\textbf{Ans: C}

\end{answer}
\end{document}
